\chapter*{Introduction}
\addcontentsline{toc}{chapter}{Introduction}

Monte-Carlo tree search turned out as an algorithm reaching unprecedent results in playing a
game of Go \cite{Chaslot2008} and variety of other games. Accordingly to properties of the
algorithm, Monte-Carlo tree search has been suggested as an algorithm suitable for canonical 
multi-robot
application. In our thesis, we aim on design and experimental evaluation of algorithms based on
Monte-Carlo tree search suitable for the canonical multi-robot application.

For simplicity, we consider a game with a team of cooperative players as an environment
simulating canonical multi-robot application. But in the cooperation
of the players, we respect nature of real-life problems, such as limited communication,
transmission delays or communication failures. On the other hand, we work with fully
observable discrete environment what .

Initial point of our research was review of parallelization approaches to Monte-Carlo tree 
search \cites{Cazenave2007}{Chaslot2008}{Teytaud2008} what gave us a solid cornerstone for the
further work. Second source of initial inspiration, especially for discussion on communication
between agents, were distributed constraint optimization algorithms \cite{Zivan2009}. Finally,
the third influence for the thesis were articles dealing with application of Monte-Carlo tree
search to a domain of a game of Ms Pac-Man \cites{Ikehata2011}{Nguyen2011}.

The thesis is divided into four chapters. Chapter \ref{chap_mas} contains general notes on games 
with team
of cooperative agents, including definition of such games and discussion on inter-agent
communication. 

In Chapter \ref{chap_mcts}, relevant research on Monte-Carlo tree search is reviewed. 
In particular, the algorithm itself is described in Section \ref{sec_mcts_description}
\todo{konvergence, mcts pro 2 hrace?} and approaches to parallelization of Monte-Carlo tree
search are reviewed in Section \ref{sec_parallel_mcts}. 

Once Monte-Carlo tree search is
reviewed and games with teams are defined, we propose our distributed algorithms Chapter
\ref{chap_dmcts_design}. Next to simple algorithms serving for comparison (independent agents,
joint-action exchanging agents), we propose two algorithms based on existing parallel
algorithms (root exchanging agents, simulation results exchanging agents) and one algorithm
in which we attempted to create an algorithm having good properties of previous two algorithm
but suppressing bad ones. General properties and comparison of the algorithms are discussed 
in Section \ref{sec_dmcts_comparison}.

Chapter \ref{chap_evaluation} contains evaluation of proposed algorithms on a domain of a
game of Ms Pac-Man with usage of Ms Pac-Man vs Ghosts framework \cite{PacmanVsGhosts}. In
particular, we describe the framework in Section \ref{sec_pacman_vs_ghosts} together with rules
of the game and simplifications of the game we have done for our purpuses. We give relevant
notes on implementation of Monte-Carlo tree search on the chosen domain and simulation of
distributed environment with real-world properties (communication failures etc.) in Section 
\ref{sec_implementation_notes}. Before we have run evaluational experiments, we devoted some time for
basic tunings of parameters of Monte-Carlo tree search (Section \ref{sec_mcts_tuning}).
Finally, we propose results of the experiments together with discussion on them and comparison
of the algorithms in terms of the results in Section \ref{sec_dmcts_experiments_comparison}.

