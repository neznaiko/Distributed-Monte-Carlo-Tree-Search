\chapter*{Conclusion}
\addcontentsline{toc}{chapter}{Conclusion}


\section*{Summary}
\addcontentsline{toc}{section}{Summary}

This work investigates possibilities of usage of Monte-Carlo tree search for distributed
multi-agent 
coordination. Specially we have considered the problems of inter-agent information
transmission, such as communication failures.

We have reviewed the
perfect-information game and proposed the definition enriching such game with teams.
Additionally, we have reviewed general coordination and communication issues together with
approaches to the distributed team coordination inspired with distributed constraints
optimization problems. One chapter is devoted to description of Monte-Carlo tree search and its
variant (UCT) and to existing MCTS parallelization algorithms.

Main result is a set of distributed algorithms based on MCTS dedicated to action planning in
the environment of perfect-information games with teams. We have designed several algorithms of
various levels of complexity, communication requirements and scalability trade-offs. We have
put detailed description of the design of the proposed algorithms. 

For the
evaluation, we have used the domain of a simplified game of Ms Pac-Man. It provided us testing 
environment with four cooperative agents. The quality of particular algorithms have been
measured in dependence on computational time, amount of communication and communication
failures. In terms of the properties, we have compiled a comparison of the algorithms and
suggested the best candidates suitable for the solution of team coordination problem.
\emph{Root exchanging agents} algorithm, based on existing parallel MCTS algorithm,
showed up as the best of such candidates. It reached 
the overall best
performace, with small communication requirements and good robustness against communication
failures. The second adaptation of existing parallel algorithms is \emph{simulation results
passing agents} which, in opposite to root exchanging agents, performed worst (if we don't
consider \emph{independent agents} algorithm proposed for comparison purposes) and requiring
the highest amount of communication. The algorithm has still some good general properties such
as deeper insight of the tree so we developed third algorithm as a trial of integration of good
properties of previous two algorithms - \emph{tree-cut exchanging agents}. The algorithm
requires more communication than root exchanging agents and performs bit worse, yet some open 
questions of the algorithm design retains (see Future Work). The last algorithm,
\emph{joint-action exchanging agents}, showed unexpectedly good results. With its simplicity
and very low communication requirements and good robustness, it outperformed simulation results 
passing algorithm.

Part of this thesis is also DVD containing source codes of implemented algorithms, experimental
data, source codes of this thesis and other tools used for performing experiments (see
Attachment 1).


\section*{Future Work}
\addcontentsline{toc}{section}{Future Work}

\begin{enumerate}
\item The algorithms can be used for coordination of large-scale
teams. Scalability of algorithm proposed in our thesis is limited since
agents' trees contain information. If the tree would contain information reduced to some
neighbourhood of the agent, it would lead to better scalability. Algorithms solving distributed 
constraints optimization problems could also bring new ideas to improvement of the scalability.

\item Algorithm of \emph{tree-cut exchanging agents} can be further enquired. Especially, (1)
tuning of number of cuts transmitted per a game step may be tuned and (2) the way of
transmission of the cuts may be adjusted. In original algorithm, cuts are transmitted in one
message so if the communication failure occurs in the middle of the message, whole cut is lost.
But cuts may be transmitted node by node what would lead to better robustness of the algorithm.
(3) The algorithm has also a special property that its performance starts decreasing with
certain amount of communication (what caused by too much redundant communication leading to
unadequate repetitiveness of applying of the almost same tree cuts). This property may be
also further enquired and some dynamic balancing of the communication may be performed by the
algorithm to avoid decrease of the performance.

\end{enumerate}


