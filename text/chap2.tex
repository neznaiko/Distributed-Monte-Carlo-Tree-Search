\chapter{Games with Team of Cooperative Agents}
\label{chap_mas}

\todo{Název kapitoly, koncepce}

\todo{Extraction from the related chapters from \ref{MAS2008}}

\todo{Neúplné algoritmy z DCOP článků}


\section{Perfect-information game}

\todo{Převzato z MAS, doprovodný text}

\todo{Sladit značení a terminologii ve zbytku textu}

This thesis deals with the problem of coordination of agents in games. Before we will review
work on this topic and propose our algorithms, it is necessary to define what the game in our
context is. We will consider finite, discrete, full observable, deterministic and static
environment in which we will talk about \emph{perfect-information game}. Following definition
is based on the definition
proposed in \cite{MAS2008} and generalized for our purposes (original definition does not
include posibility of simultaneous play of multiple players and forces a play of exactly one
player).


\newtheorem*{defgpig}{Definition}
\begin{defgpig}[Perfect-information game]

A (finite) \textbf{perfect-information game} (in extensive form) is a tuple $G =
(P,A\cup\{\lambda\},S=S_n\cup S_p,\chi,\rho,\sigma,u)$, where:

\begin{itemize}

\item $P$ is a set of $n$ players;

\item $A$ is a (single) set of actions;

\item $S_n$ is a set of nonterminal choice states;

\item $S_t$ is a set of terminal states, disjoint from $S_n$;

\item $\chi: S_n \times P \mapsto 2^A \cup \{\lambda\} \setminus \varnothing$ is the action 
 function, which assigns to each choice node and player a set of possible actions, $\lambda \notin
 A$ is a neutral move played by the players not being on turn;

\item $\sigma: \{(s,a)| s \in S_n, a \in \prod\limits_{i\in P}\chi(s,i)\} \mapsto S$ is the
successor function, which maps a choice node and an action tuple to a new choice node or
terminal node such that for all $s_1, s_2 \in S_n$ and $a_1 \in \prod\limits_{i\in
P}\chi(s_1,i), a_2 \in \prod\limits_{i\in P}\chi(s_2,i)$, if 
$\sigma(s_1,a_1) = \sigma(s_2,a_2)$ then $s_1=s_2$ and $a_1=a_2$; and

\item $u = (u_1,\ldots,u_n)$, where $u_i: S_t \mapsto \mathbb{R}$ is a real-valued utility
function for player $i$ on the terminal nodes $S_t$.

\end{itemize}

We say that a perfect-information game is \textbf{turn-based} if each turn at most one player plays a
turn, in other words, if the following restriction on action function $\chi$ is satisfied:

\begin{equation}
\label{eq_turn_based_game}
\forall s \in S_n |\{i \in P|\chi(s,i) \not= \{\lambda\}\}| \le 1
\end{equation}

Contrariwise we will call a perfect-information game \textbf{simultaneous} if simultaneous
plays of multiple players are permitted and a state forcing a play of at least two players
exists (and so Equation \ref{eq_turn_based_game} is not satisfied).

\end{defgpig}

Because this thesis works only with perfect-information games, we will usually refer to
perfect-information game and its variants as to \emph{game}, \emph{turn-based game} and
\emph{simultaneous game}. We will also bound a utility function to $[0,1]$.

The aim of a player $i$ is reaching a terminal node $s$ with the greatest possible utility
$u_i(s)$.


\section{Team Coordination}

\subsection{Games with Teams of Players}

Definition of perferct-information game serves (without further modifications) as an
environment for \emph{teams of players}. A team of players (or simply a team) is a set of agents
attempting to reach the best result possible collectively. In other words, agents of a team
shares the utility. 

This can be simply applied to a perfect-information game. A team can be represented as a single
player of a perfect-information game where actions of such a player is joint-action composed of
all actions played by team members. The second possibility how to model a team in
perfect-information game is simply applying the definition of a team and considering players
with the same utility.

\todo{Rozdíl mezi přístupy v předchozím odstavci, komunikace, ...}


\subsection{Communication}
\subsection{Distributed Constraints Optimization (??)}

\section{Conclusion}
